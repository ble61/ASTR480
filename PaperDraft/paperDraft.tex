\documentclass[12pt]{article}
\usepackage{}
\usepackage[margin=2cm]{geometry}
\usepackage[authoryear,comma,round]{natbib}
\usepackage[separate-uncertainty=true,multi-part-units=single]{siunitx}
\usepackage{datatool,booktabs,hyperref,graphicx,caption,amsmath,amssymb}
\usepackage[list=true,listformat=simple]{subcaption}
\usepackage[para]{footmisc}
\usepackage[super]{nth}
\usepackage[utf8]{inputenc}

\DeclareSIUnit{\arcsec}{''}
\DeclareSIUnit{\au}{\mathrm{AU}}
\DeclareSIUnit{\mag}{\mathrm{mag}}
\DeclareSIUnit{\px}{\mathrm{px}}

\newcommand{\ttt}{\texttt}
\newcommand{\tess}{\textit{TESS}}
\newcommand{\tessellate}{\texttt{TESSELLATE}}
\newcommand{\red}[1]{\textcolor{red}{#1}}

\title{TESSELLATE asteroids paper}
\author{Brayden Leicester, Ryan Ridden-Harper, Michele Bannister and Others}
\begin{document}
\maketitle

\section{Abstract}\label{sec:Abs}
%TODO Later

% \twocolumn %To check size


\section{Introduction}\label{sec:Intro}

%TODO general opening statment about asteroids as transiet events/ moving objects

Since 2017, the Transiting Exoplanet Survey Satellite \citep[\tess,][]{Ricker2014} has been obseving large swathes ($\qty{96}{\degree}\times\qty{24}{\degree}$) of sky at a very high cadence.
These month long blocks of full frame images (FFIs) are named sectors, with the cadence of the sectors dependant on the cycle.
The cadence started at \qty{30}{\minute}, then increased to \qty{10}{\minute}, and has now increased again to \qty{200}{\second} FFIs.

The large archival dataset of \tess\ has gone largely unanalysed.
This is until the \tessellate\ project \defcitealias{TESSELLATE}{H. Roxburgh and R. Ridden-Harper et al. 2025}\citepalias{TESSELLATE}, which seeks to find all the transient events that were serendipitously observed by \tess.
\tessellate\ searches every pixel of every FFI from \tess, using the \ttt{TESSreduce} \citep{Ridden-Harper2021a} to looking for changes in brightness that could be related to a physical event.

Asteroids are the most numerous source of non-transient detections in \tessellate, and need to be filtered out of the rest of the data.
This also allows for analysis of the properties of these planetesimals.
This letter presents our analysis of the asteroids found with the first set of \tessellate\ sectors, with a \qty{10}{\minute} cadence.
This cadence is fast enough that the Nyquist frequency of the observations is above the spin barrier for asteroids \citep{Pravec2000}, so \tess\ could detect new fast rotators.

There have been other studies done using \tess\ to characterise asteroids.
It was suggested by \citet{Pal2018} that it would be possible to get good photometry down to $V \lesssim \qty{19}{\mag}$.
They followed up with an analysis of the small bodies the first 12 sectors (cycle 1) \citep{Pal2020}, tripleing the number of asteroids with accurate rotation periods.
\citet{McNeill2023} re-analyses the same sectors of cycle 1, and is in broad agreement with \citeauthor{Pal2020}.

\tess\ is used for solar-system observation in many other ways aswell.
The Minor Planet Center (MPC) gets regular updates on asteroids in \tess\ from the \ttt{LINEAR-TESS} program \citep{Woods2021}.
\citet{Gowanlock2024} uses \tess\ photometry as well as observations by the Zwicky Transient Facility \citep[ZTF, ][]{Bellm2019} to get a longer baseline on mutually observed objects while combining ground and space-based observations.
Single object analysis is also possible, \citet{Humes2024} recover the period of an asteroid commensurate with Earth's rotation.
They also use the \tess\ data alongside other observations to determine the shape of the object.
\tess\ has been used to study comets as well \citep[e.g.][]{Ridden-Harper2021b}
Fainter and unknown solar system objects can be found by shift and stacking \citep{Holman2019, Payne2019, Rice2020} or taking a fast X-ray transform \citep{Nguyen2024} of the \tess\ FFI data.

For our analysis of the asteroids in \tess, we use the third year of observations from the spacecraft; sectors 27 to 39. 
These were the first sectors to have a \qty{10}{\minute} cadence.
Our methods are presented in \autoref{sec:Meth}, followed by the results in \autoref{sec:Res}.
A discussion of how our results compare to other \tess\ asteroid studies is given in \autoref{sec:Dis}, and we conclude in \autoref{sec:Conc}.

\section{Methods}\label{sec:Meth}

%-How they are found in TESSELLATE accidentally

\tessellate\ provides a new way to find and characterize asteroids: detecting them as transient events.
The lightcurves of these detections often have spike only a few frames long, as the asteroid moves over the pixel of interest.
Asteroids move at about \qty{1}{\px} per frame for the \qty{30}{\min} cadence \citep{Pal2018}, and thus are found as many different transient events as they cross a sector.
Both \ttt{SourceDetect} and \ttt{Starfind}, the detection methods employed by \tessellate, detect planetesimals.

Characterizing the spikes in the single pixel lightcurves as an ``asteroid'' has it's problems, as real transient events can look similar.
We have implemented another way of finding and removing asteroids from the set of detected events.
The reduction of full sectors during a \tessellate\ run also allows for forced photometry at known asteroid positions.
These can then be matched with the detections to classify the spikes, and properties of the asteroids can be measured at the same time.

%-Conesearch w/ SkyBoT and names w/ Horizons

The position of known asteroids is important for matching them to detections.
We use \ttt{SkyBoT} \citep{Berthier2006} conesearch to find all the asteroids with $V\leq \qty{20}{\mag}$, the limit suggested by \citep{Pal2018}.
We query every \qty{12}{\hour} over the month of the sector, and interpolate the positions to the times of the \tess\ observations.
With \tess's large pixel scale, these interpolations are accurate enough for our purposes, and save on time and resources, as the query can be expensive.
%? They were checked for accuracy against JPL Horizons

%-Matching to TESSELLATE
The positions of these asteroids can be matched with the detections, using a KDTree \citep{Maneewongvatana1999}.
Setting boundaries of \qty{0.1}{\day} and \qty{1}{\px} %TODO check it is 1 px.
on the nearest match keeps only objects that are coincident temporally and spatially.
If two such objects  match to the same detection, the shortest distance between the positions is taken.
The matched events are then classified as asteroids, and are not considered further by \tessellate.%Clunky end to sentance

%-Forced Photometry Lightcurve

Forced photometry along the tracks of the asteroids allows for the construction of lightcurves.
The differenced images are used for the forced photometry, as the objects are moving they are not dimmed by the differencing. 
% The positions are readjusted to the center of mass (COM) of the object.
Aperture photometry, using \texttt{Photutils} \citep{Bradley2024}, is preformed at the centre of mass of the point spread function with a \qty{1.5}{\px} aperture, the standard size of an aperture in \tessellate.
The lightcurves are then sigma-clipped to $3\sigma$ from the mean remove background stellar contamination.


%-Lomb-Scargle Periodogram

From their lightcurves, the rotation period of the object can be determined by using a Lomb-Scargle periodogram \citep[\citet{Lomb1976, Scargle1982}, but see][ for a review]{VanderPlas2018}.
The new \ttt{NIFTY-LS} package \citep{Garrison2024},  as implemented in \ttt{astropy} \citep{Astropy2013,Astropy2018,Astropy2022}, was used to calculate the periodogram.
The periodogram searched for rotation periods between the Nyquist limit of \qty{20}{\minute} and a maximum of \qty{17}{\day} \citep[the value used in][ due to the length of the lightcurve]{McNeill2023}.

-Example Figure

An example of the periodogram fit to the lightcurve is given by \autoref{fig:perEx}. 
A clear maximum power is found in the top panel, and the associated frequency is fed to the model shown in the middle panel.
The residuals in the lower panel are small for the most part, with large deviations from 0 only where the lightcurve has large uncertainty in its flux. 
Thanks to the known frequency from LCDB plotted, we note that we recover the double frequency alias, this is explored further below.   

-Quality Checks on the lightcurves

Not every lightcurve and periodogram is good enough for analysis, so some quality cuts are applied.
A minimum brightness cut of 10 counts $\sim \qty{18}{mag}$ %?which way around is better? 
is applied, otherwise the lightcurve is indistinguishable from noise. %* This is emperical from match to LCDB, should that fig be here?
The periodograms could return the \qty{17}{\day} maximum value if no physical period was detected, so following \citet{McNeill2023} we discard any periods within \qty{10}{\percent} of this maximum. %TODO min period of \qty{1}{\hour} 
%! This and the doubling of the periods means we can't get fast rotators at all. Need to think this through
Also following \citeauthor{McNeill2023}, we do not trust the periods of any lightcurve with $\leq 200$ datapoints.
%TODO are there more checks

%-Run all these are part of TESSELLATE
The above methods are integrated into the rest of the \tessellate\ pipeline. %*I should probably get around to doing that...
This allows for asteroids to be filtered out of the new detections as they happen.
The results given below will be improved upon as more sectors are run and a higher number of minor planets can have their rotation properties measured.


\section{Results}\label{sec:Res}

-Total Asteroids found

\red{A Number \#\#\#} of asteroids with $V\leq \qty{20}{\mag}$ were found in the sectors we analysed.
Most of these are too faint for the lightcurve to give a reliable and period.
Once the quality checks described above have been applied, we are left with \red{A smaller number \#\#\#} of planetesimals with accurate rotation rates.
\red{Some fraction} of these asteroids are not in the LCDB,%TODO check defind before
so we are the first to get the rotation period for these bodies.

-Properties of those that pass

%TODO table of asteroids by class and frac that pass.

\section{Discussion} \label{sec:Dis}

-Comp to Pal and McNeill

-Fast Rotator Plot

We do not find any fast rotators in our dataset.
While our Nyquist limit is well above the spin barrier of \citet{Pravec2000} thanks to \tess's fast cadence, most asteroids discovered above the barrier are small%TODO CITE Warner2009 LCDB?
, and therefore dim.
The shallow magnitude of \tessellate\ \citep{TESSELLATE} makes it difficult for dim planetesimals to pass the quality checks.

%! we also can't actually go below 2hrs even with a 10 min cadence because we double the period.

-other discussion

The rotation periods of the brightest asteroids are not recovered.
This is due to the fixed aperture size of our forced photometry, \qty{1.5}{\px}.
The bright asteroids can start to saturate the \tess\ detector, and can bloom out to occupy more pixels.
We chose to keep a constant aperture to standardise the analysis, and to be agnostic to the properties of the asteroid being analysed.

Some asteroids slip through our detection methods, and end up unclassified in the \tessellate\ data.
This is often because the detections of bright asteroids are offset from the predicted positions by further than the matching radius we set.
Asteroids are easy to spot for our Cosmic Cataclysms \ttt{Zooniverse}\footnote{\url{https://www.zooniverse.org/projects/cheerfuluser/cosmic-cataclysms}} volunteers, as they are at a different position in the ``1 hour later'' panel, so any that slip past can be caught here.


\section{Conclusion}\label{sec:Conc}
%TODO Later

\bibliographystyle{aasjournal}
\bibliography{paperBib.bib}

\end{document}