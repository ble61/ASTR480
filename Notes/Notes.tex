\documentclass[12pt]{article}
\usepackage{natbib, hyperref,graphicx}

\usepackage[separate-uncertainty=true,multi-part-units=single]{siunitx}

\DeclareSIUnit{\arcsec}{''}

\usepackage[utf8]{inputenc}
\usepackage{newunicodechar}
%\usepackage{libertine}

\DeclareRobustCommand{\okina}{%
  \raisebox{\dimexpr\fontcharht\font`A-\height}{%
    \scalebox{0.8}{`}%
  }%
}
\newunicodechar{ʻ}{\okina}

\newcommand{\omuamua}{\okina Omuamua}


\title{A480 Notes}
\author{Brayden Leicester}
\begin{document}
\maketitle

\section{Things to Check}
The named asteroids in \cite{Harris1994} (most notably 4179 Toutatis) should be checked for TESS observations. As these should tumble.

288 Glauke \cite{Harris2015}

\section{Background writing}

%TODO TESS intro
The Transiting Exoplanet Survey Satellite (TESS) \cite{Ricker2014} is a large area, high imaging cadence, space telescope. 
TESS is tasked with staring at the same piece of sky for \qty{27}{\day} at a time (a sector), delivering \qtyproduct{96 x 24}{\degree} full frame images (FFIs) at regular intervals. 
These FFIs are built from stacked \qty{2}{\second} exposures, leveraging the short readout times of the 16 CCD cameras onboard. 
With the initial cadence for these full frame images set to \qty{30}{\minute}, the time resolution of TESS is unparalleled. 
This does come at the cost of spatial resolution, as the pixels are each \qty{21}{\arcsec} square. 
A Nyquist frequency of \unit{\per\hour} is well sampled enough to characterise most variable stars, as well as find the orbital periods of exoplanets to a high precision. 
After the initial mission had mapped the entire sky, TESS started to reduce the time of the full frame images, down to \qty{10}{\minute} and then down to only \qty{200}{\second}.

Of interest here is what such a high sampling rate can do for statistic on the asteroid population. 
For bright asteroids the rotation periods should be able to be easily determined from this vast dataset. 
The shortest FFIs will be able to accurately determine the rotation periods of all but the fastest rotating asteroids, most of which will be too dim to see in the TESS data. 
There have been attempts before to find and classify the asteroids in TESS data before by \citet{Pal2018, Pal2020}. This work aims to extend there study to more sectors, and a to use a different data reduction method, using the \texttt{TESSreduce} package \citep{Ridden-Harper2021}. 
As part of a full sky transient survey using TESS, \texttt{TESSELLATE} (Ridden-Harper and Roxburgh et al., in prep), asteroids pop out as transient objects that are spikes in brightness of pixels for only a few frames. 
They can confuse pipelines looking for short time stellar transients such as flare stars and supernova. 
As such, a way of filtering them out is required, as well as this filtering will get the properties of the lightcurves, such as the rotation periods and amplitude variation of these small solar system bodies.
A full sky, self consistent catalogue of these properties is important for many reasons. 
The asteroid population statistics are useful in their own right, understanding the orbits of near earth objects is useful for planetary defence, and these bodies \dots      

%TODO 1I observation paragraph/s 

The first interstellar object (ISO) was discovered in late 2017, it has come to be called 1I/\omuamua \citep[see][for a review]{Bannister2019}.
\omuamua was determined to be spectroscopically red \citep{Fitzsimmons2017, Meech2017}, and having a photometric colour in the neutral end of the solar system range \citep{Bannister2017}. 
It appeared to be tumbling \citep[e.g.][]{Drahus2018,Fraser2018}, the double peaked light curve was found to have a rotation period of \qty{8.67(0.34)}{\hour} \citep{Belton2018}.

Peak to peak LC of 2.5 mag %TODO cite

Combing the tumbling with an elongation ratio of up to \qty{6(1)}{}:1 \citep{McNeill2018}, 1I is said to have a cigar shape \citep{Belton2018}.  


1I was classed as asteroid due to lack of a coma, and no noticable activity %TODO cite


\section{Method Notes}

%*Pos gathering
To check for asteroids in the TESS data, the positions of the asteroids with time are required.
For most asteroids, their orbital elements are well known, so it is a matter of looking them up and cross-matching with transients in the TESS data.
However, querying APIs for timesteps of \qty{30}{\min} or shorter is
prohibitively expensive, especially when they rate limit their calls.
Python was used to make API calls to {Skybot}\footnote{\href{https://vo.imcce.fr/webservices/skybot/}{Skybot}} to get positions of asteroids in a cone search box in RA and Dec space.
As TESS sectors are 27%?Check
days long, querying every \qty{12}{\hour} is manageable.
These positions are still very sparsely spaced in time compared to the TESS data, so an interpolation is needed to bridge the gap.
With TESS data coming in $\frac12\unit{\hour}$ chunks, 24 interpolated points are needed between each API call.
Assuming this is fine is justified, as asteroids move at close to a TESS pixel per TESS frame \citep{Pal2018,Pal2020}.
For the faster TESS data, more interpolated points are needed, but a smaller the change in position between each point.


In another series of API calls, this time querying {JPL Horizons}\footnote{\href{https://ssd.jpl.nasa.gov/horizons/}{JPL Horizons}} by object name, the orbital elements of the asteroids can be found.
From these elements the type of asteroids (i.e. main belt, NEOs, Jupiter Trojan etc.) can be determined by plotting the semi-major axis, $a$, against the inclination, $i$, and the eccentricity, $e$, of the body.
Osculating elements, such as the heliocentric distance, $r_h$ and the distance to the observer, delta $\delta$%?check
can also be found from Horizons.
Plotting inclination against a change in these distances with time can give an idea of the motion of the bodies.
The JPL queries can also be used to check the validity of the interpolation from Skybot.
By geting the RA and Dec values of a named asteroid for all the timesteps interpolated, a $\Delta$RA against $\Delta$Dec plot can be made.
This was heavily rate limited, as the epoch querying did not want to play ball with date formats, but what was produced showed that the interpolation was accurate to within an arcsecond most of the time, the earliest points were the most astray, but this was still within three or four arcseconds, much smaller than the \qty{21}{\arcsec} TESS pixels.

%*Interpolation

%*Cross matching
Having interpolated these positions, there was a well sampled set of RA, Dec and time values of where these asteroids should be in the TESS data.
They should show up in a pixel for a small number of frames, of order $\sim1$.
This is the same as a lot of other transient events, a sharp rise in brightness and then disapering quickly again.
The number of frames do change, type Ia supernova will brighten is a matter of a few hours and then dim for days, while stellar flares are of similar profile by a smaller max brightness and a corrsepondingly shorter decay time.
Asteroids are very short, however detection pipelines are robust.
These pipelines have already found the aforementioned supernova and stellar flares, the job of this work is to catch all the asteroids in the set of all the transients.
To do this, catelogue matching is in order.



%*Period fit

\bibliographystyle{apalike}
\bibliography{bibfile.bib}

\end{document}