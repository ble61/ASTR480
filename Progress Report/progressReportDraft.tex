\documentclass[12pt]{article}

\usepackage{natbib,hyperref,graphicx}
\usepackage[separate-uncertainty=true,multi-part-units=single]{siunitx}

\usepackage[margin=1.5cm, left=3cm,twoside]{geometry}



\DeclareSIUnit{\arcsec}{''}

\usepackage[utf8]{inputenc}
\usepackage{newunicodechar}
%\usepackage{libertine}

\DeclareRobustCommand{\okina}{%
  \raisebox{\dimexpr\fontcharht\font`A-\height}{%
    \scalebox{0.8}{`}%
  }%
}
\newunicodechar{ʻ}{\okina}
\newcommand{\omuamua}{\okina Omuamua }


\title{ASTR480 Progress Report}
\author{Brayden Leicester \\[1ex]
\small{Supervisors: Michele Bannister and Ryan Ridden-Harper}
}
\begin{document}
\maketitle

%*Aim / background summary

%To get all asteriods in TESS. 
This project aims to find and characterise the lightcurves of all the asteroids seen by the Transiting Exoplanet Survey Satellite (TESS). 
\section{Background}
    %Why TESS
TESS is a large area, high imaging cadence, space telescope  \citep{Ricker2014}. 
TESS is tasked with observing one piece of sky for \qty{27}{\day} at a time (a sector), delivering \qtyproduct{96 x 24}{\degree} full frame images (FFIs) at regular intervals. 
With the initial cadence for these full frame images set to \qty{30}{\minute}, the time resolution of TESS is unparalleled, with a Nyquist frequency of \qty{1}{\per\hour}, as the mission was extended the lenght of the FFIs has come down to \qty{10}{\minute} and then \qty{200}{\second}. 
This high time resolution and observation area does come at the cost of spatial resolution, as the pixels are each \qty{21}{\arcsec} square.
There have been attempts before to find and classify the asteroids in TESS data before by \citet{Pal2018, Pal2020}. This work aims to extend their study to more sectors, and to use a different data reduction method, the \texttt{TESSreduce} package \citep{Ridden-Harper2021}.
Because of the survey properties, TESS provides a selfconsitent was to measure the properties of asteroids over the full sky.   
Another beneficial part of this work is that as part of a full sky transient survey using TESS, \texttt{TESSELLATE} (Ridden-Harper and Roxburgh et al., in prep), asteroids are transient objects that spike the brightness of a pixel for only a few frames.
The goal of finding all the asteroids will allow for the removal of these spikes from the transient pipeline, as well as to understand the asteroid population better.  


    %Why Asteroids
Asteroids are a key class of solar system objects. 
Understanding their rotation properties has long been of interest to astronomers \citep[e.g.][]{Weidenschilling1981,Harris1994}.

        %Omuamua
High amplitude variation has come to the forefront of questions about asteroid properties because of the first interstellar object (ISO) 1I/\omuamua \citep[see][for a review]{Bannister2019}. 
\omuamua was measured to have a rotation period of \qty{8.67(0.34)}{\hour} \citep{Belton2018} and seemed to be tumbling \citep[e.g.][]{Drahus2018,Fraser2018}. 
The peak to peak amplitude variation of 2.5 mag %TODO cite
on the double peaked light curve is of interest, as this is much higher than most asteroids. 
With the full sky survey of bright asteroids, we hope to find many asteroids with such a large amplitude variation, and to see just how rare \omuamua is.  

\section{Work So Far}
%* Method / Initial Results

%Querry and interpolate


%Match to detections
Matching these interpolated positions to \texttt{TESSELLATE} detections is important to lower the unknown transient outputs of this pipeline. 
Using the \texttt{KDTree} algorithm \citep{Maneewongvatana1999} as implemented in \texttt{SciPy} \citep{2020SciPy-NMeth}, the right ascension (RA) and  declination (Dec) coordinates of the interpolated points and the detections can be compared and matched together. 
Filtering this \texttt{KDTree} output by not allowing the time between spatialy coincident matches to be longer than \qty{0.1}{\day}, \dots 

%Lightcurves; detected VS forced interpolated 

\section{Future Work}
%* Future work

%Periods and amplitueds for everything detected

%Run on server
The \texttt{TESSELLATE} pipeline has been running on the OzSTAR supercomputing facilities. 
After I am confident that all the parts of the asteriod detection and subsequent lightcurve analysis works as required, the same code can be refactored to work on OzSTAR and a largescale analysis of all of the processed TESS sectors can be run. 
Only after this has completed can the asteriod population statistics can be computed. 


\section{Figures}
%*Figures:

%*Bib
\def\bibfont{\tiny}
\bibliographystyle{jphysicsB}
\bibliography{../Notes/bibfile.bib}
\end{document}



