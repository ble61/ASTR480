\documentclass[12pt]{article}

\usepackage{natbib,hyperref,graphicx}
\usepackage[separate-uncertainty=true,multi-part-units=single]{siunitx}

\usepackage[margin=1.5cm, left=3cm,twoside]{geometry}

\usepackage[para]{footmisc}


\usepackage[compact]{titlesec}
\titlespacing{\section}{0pt}{2ex}{1ex}
\titlespacing{\subsection}{0pt}{1ex}{0ex}
\titlespacing{\subsubsection}{0pt}{0.5ex}{0ex}

\DeclareSIUnit{\arcsec}{''}

\usepackage[utf8]{inputenc}
\usepackage{newunicodechar}
%\usepackage{libertine}

\DeclareRobustCommand{\okina}{%
  \raisebox{\dimexpr\fontcharht\font`A-\height}{%
    \scalebox{0.8}{`}%
  }%
}
\newunicodechar{ʻ}{\okina}
\newcommand{\omuamua}{\okina Omuamua }


\title{ASTR480 Progress Report}
\author{Brayden Leicester \\[1ex]
\small{Supervisors: Michele Bannister and Ryan Ridden-Harper}
}
\begin{document}
\maketitle

%*Aim / background summary

%To get all asteriods in TESS. 
This project aims to find and characterise the lightcurves of all the asteroids seen by the Transiting Exoplanet Survey Satellite (TESS). 
\section{Background}
    %Why TESS
TESS is a large area, high imaging cadence, space telescope  \citep{Ricker2014}. 
TESS is tasked with observing one piece of sky for \qty{27}{\day} at a time (a sector), delivering \qtyproduct{96 x 24}{\degree} full frame images (FFIs) at regular intervals. 
With the initial cadence for these full frame images set to \qty{30}{\minute}, the time resolution of TESS is unparalleled, with a Nyquist frequency of \qty{1}{\per\hour}, as the mission was extended the lenght of the FFIs has come down to \qty{10}{\minute} and then \qty{200}{\second}. 
This high time resolution and observation area does come at the cost of spatial resolution, as the pixels are each \qty{21}{\arcsec} square.
There have been attempts before to find and classify the asteroids in TESS data before by \citet{Pal2018, Pal2020}. This work aims to extend their study to more sectors, and to use a different data reduction method, the \texttt{TESSreduce} package \citep{Ridden-Harper2021}.
Because of the survey properties, TESS provides a selfconsitent was to measure the properties of asteroids over the full sky.   
Another beneficial part of this work is that as part of a full sky transient survey using TESS, \texttt{TESSELLATE} (Ridden-Harper and Roxburgh et al., in prep), asteroids are transient objects that spike the brightness of a pixel for only a few frames.
The goal of finding all the asteroids will allow for the removal of these spikes from the transient pipeline, as well as to understand the asteroid population better.  


    %Why Asteroids
Asteroids are a key class of solar system objects. 
Understanding their rotation properties has long been of interest to astronomers \citep[e.g.][]{Weidenschilling1981,Harris1994}. %Omuamu
High amplitude variation has come to the forefront of questions about asteroid properties because of the first interstellar object (ISO) 1I/\omuamua \citep[see][for a review]{Bannister2019}. 
\omuamua was measured to have a rotation period of \qty{8.67(0.34)}{\hour} \citep{Belton2018} and seemed to be tumbling \citep[e.g.][]{Drahus2018,Fraser2018}. 
The peak to peak amplitude variation of 2.5 mag \citep{Meech2017} of the double peaked light curve is of interest, as this is much higher than most asteroids, and it implies a large axis ratio. 
With the full sky survey of bright asteroids, we hope to find many asteroids with such a large amplitude variation, and to see just how rare \omuamua is.  

\section{Work So Far}
%* Method / Initial Results

%Querry and interpolate
To check for asteroids in the TESS data, the positions of the asteroids with time are required.
For most asteroids, their orbital elements are well known, so it is a matter of looking them up and cross-matching with transients in the TESS data.
Python was used to make API calls to {Skybot}\footnote{\href{https://vo.imcce.fr/webservices/skybot/}{Skybot}} to get positions of asteroids in a cone search box in RA and Dec space.
As TESS sectors are \qty{27}{\day} long, querying every \qty{12}{\hour} is manageable.
These positions are still very sparsely spaced in time compared to the TESS data, so an interpolation is needed to bridge the gap.
With TESS data coming in $\frac12\unit{\hour}$ chunks, 24 interpolated points are needed between each API call.
This interpolation should be accurate, as asteroids move at close to a TESS pixel per TESS frame on average \citep{Pal2018,Pal2020}, and checking against a higher frequency query to {JPL Horizons}\footnote{\href{https://ssd.jpl.nasa.gov/horizons/}{JPL Horizons}} confirmed this.
For the faster TESS data, more interpolated points are needed, but a smaller the change in position between each point.

%Match to detections
Matching these interpolated positions to \texttt{TESSELLATE} detections is important to lower the unknown transient outputs of this pipeline. 
Using the \texttt{KDTree} algorithm \citep{Maneewongvatana1999} as implemented in \texttt{SciPy} \citep{2020SciPy-NMeth}, the right ascension (RA) and  declination (Dec) coordinates of the interpolated points and the detections can be compared and matched together. 
Filtering this \texttt{KDTree} output by not allowing the time between spatially coincident matches to be longer than \qty{0.1}{\day} stops any accidental matches in position from non-asteroids. 

%Lightcurves; detected VS forced interpolated 
There are two sets of points to take lightcurves from. 
The matches from the detections, which already have a flux calculated, or the interpolated points themselves, which are more numerous but require forcing the photometry. 
Not every interpolated point gets a match, due to a variety of reasons, %pass in front of stars, dip below limiting mag, bad pipeline, not extreme enough of a difference between frames 
so a comparison between the two lightcurves is interesting. 
There were some challenges getting the flux of the interpolated points, even when \texttt{TESSELLATE} had already reduced all the FFIs of interest, due to the timing of the TESS frames, but these were identified and corrected for.

\section{Future Work}
%* Future work

%Periods and amplitueds for everything detected
The next part of my analysis has to do with determining the periods and amplitudes of each asteroid's lightcurve.
For this there are a few methods I can try, using the \texttt{Lightkurve} package built for period analysis of TESS (and Kepler) data of variable stars, or peel back a layer of abstraction and use the Lomb-Scargle periodogram as implemented by \texttt{Astropy}. 
Some trialling of both methods is needed, as preliminary testing reveals of interesting similarities and differences between the packages. 


%Run on server
The \texttt{TESSELLATE} pipeline has been running on the OzSTAR supercomputing facilities. 
After I am confident that all the parts of the asteroid detection and subsequent lightcurve analysis works as required, the same code can be refactored to work on OzSTAR and a large-scale analysis of all the processed TESS sectors can be run. 
Only after this has completed can the asteroid population statistics can be computed. 
I will be looking for completeness of detections of asteroids, as well as accuracy of periods and amplitude variation.


\section{Figures}
%*Figures:

%H-scaled pos plot of higher cam/less asteroids. 
%Both Lightcurves (interp and detect)
%? another one?


%*Bib
% \newgeometry{twocolumn,margin=1.5cm, left=3cm,twoside}
% \newpage
\def\bibfont{\tiny}
\bibliographystyle{jphysicsB}
\bibliography{../Notes/bibfile.bib}
\end{document}



