\documentclass{UCreport} 



\usepackage[authoryear,comma,round]{natbib}
\usepackage{hyperref,graphicx}
\usepackage[separate-uncertainty=true,multi-part-units=single]{siunitx}

\usepackage{caption}
\usepackage[list=true,listformat=simple]{subcaption}
\makeatletter
\g@addto@macro\p@subfigure{.}
\makeatother

% \usepackage[margin=1.5cm, left=3cm ,twoside]{geometry}

\usepackage[para]{footmisc}

\usepackage[super]{nth}

\usepackage{amsmath,amssymb}

\DeclareSIUnit{\arcsec}{''}
\DeclareSIUnit{\au}{\mathrm{AU}}
\DeclareSIUnit{\mag}{\mathrm{mag}}
\DeclareSIUnit{\px}{\mathrm{px}}

\usepackage[utf8]{inputenc}
\usepackage[T1]{fontenc}
\usepackage{newunicodechar}

\DeclareRobustCommand{\okina}{%
  \raisebox{\dimexpr\fontcharht\font`A-\height}{%
    \scalebox{0.8}{`}%
  }%
}
\newunicodechar{ʻ}{\okina}
\newcommand{\omuamua}{\okina Oumuamua }


% \title{ASTR480 Final Report:\\
% \textit{Asteroids in TESS}
% }
% \author{Brayden Leicester \\ 
% \textit{61352063} \\
% [1ex]\small{Supervisors: Michele Bannister and Ryan Ridden-Harper}
% }
\begin{document}



%----------- Report information ---------

\logo{logos/UC.png}
\school{\textbf{School of Physical and Chemical Sciences}}
\course{\textbf{ASTR480}}
\student{\textbf{Brayden Leicester}}
\supervisor{Dr Michele Bannister and Dr Ryan Ridden-Harper}
\ttitle{Asteroids in TESS} %title of the file


%----------- Init -------------------

\buildmargins % display margins
\buildcover % create the front cover of the document



%------------ Declaration ----------------
\pagenumbering{roman}
\addcontentsline{toc}{section}{Declaration}
\declaration

I certify that content of this report was mostly my own work.
\\

My supervisors helped by proofreading the report and offering feedback. As well as providing useful insight into problems when I was stuck.\\

Graduate students, X and Y, helped me understand ... \\

The derivation in section 3 was taken from ... \\

\autoref{Fig:FreqVsDiam} has been constructed from data from the Light Curve Database (LCDB) \citet{Warner2009}, as cited in text. That data has being built suplimented with my own work in the corrseponding Figure %TODO

\vspace{2cm}
\begin{centering}
  \textbf{\student}\par
\end{centering}

\newpage

%------------ Abstract ----------------

\addcontentsline{toc}{section}{Abstract}
\abstract

This is where the abstract goes. \\

This is the next paragraph.

% %*Background

% %*Methods

% %*Main Results

% %*Conclusion

\newpage
%------------ Contents pages ----------------

\toc % creates the table of contents

\addcontentsline{toc}{section}{List of Figures}
\listoffigures
\newpage
\addcontentsline{toc}{section}{List of Tables}
\listoftables
\newpage

%------------ Report body ----------------
\pagenumbering{arabic}

\section{Introduction}\label{Sec:Intro}

%To get all asteroids in TESS. 
This project aims to find and characterise the light curves of all the asteroids seen by the Transiting Exoplanet Survey Satellite (TESS).
Combining knowledge of these asteroid's positions with the high cadence imaging of TESS should allow for the determination of rotation periods of these small bodies.


\subsection{Asteroids}\label{SubSec:Asteroid}

Asteroids are a key class of solar system objects.
The orbits of asteroids can be well characterised by only a few values. The semi-major axis, $a$, is the largest distance from the sun that the asteroid achieves on its elliptical orbit.
Main belt asteroids have typical $a$s of \qtyrange{2}{3.3}{\au} \citet{DeMeo2015}.
The bounds between the inner, middle and outer main belt at \qty{2.5}{\au} and \qty{2.82}{\au}, these separations are due to resonances with Jupiter (the 3:1 and 5:2 mean motion resonances mark the edges of middle main belt) and Saturn (the $\nu_6$ secular resonance cuts off the lowest $a$ edge of the inner belt) carving swathes where asteroids cannot be in stable orbits.
The eccentricity, $e$, of this orbit is the next parameter of interest. $e=0$ implies a perfectly circular orbit, $0<e<1$ are ellipses, $e=1$ give a parabolic orbit, and $e>1$ are unbound, hyperbolic orbits.
Main belt asteroids have small $e$, in the range of \qtyrange{0}{0.35}{} \citep{DeMeo2015}
Comets have larger $e$, with some long period comets having $e\approx 1$, but the typical range is \qtyrange{0.2}{0.7}{} \citep{Lewis2012}.
Interstellar objects (ISOs) have $e>1$ as they are on unbound hyperbolic trajectories.
The inclination, $i$, of asteroids is also a key orbital element, it is the angle relative to the ecliptic plane and in principle asteroids can span the whole range.
Main belt asteroids have typical inclinations of \qtyrange{0}{30}{\degree} \citep{DeMeo2015}, so the TESS cameras closer to the ecliptic plane are expected to have the most asteroids in their fields.

% // below TODO Abs mag H, and how that is size.

%TODO asteroid families
Asteroids can be grouped into families based on clustering of the proper orbital elements and similarities in their colour or spectra \citep{Nesvorny2015}.
These families are named for their largest member.
Families are thought to all originate from the same body that was destroyed by a collision earlier in the solar system's history.

%TODO YORP spin up
Asteroids can have their rotation properties changed with time, collisions can do this, but so can light from the sun.
The YORP effect (named for the discoverers Yarkovsky, O'Keefe, Radzievskii, and Paddack) is the process of tourquing asteroids of an asymmetric shape by both bombardment of photons and thermal (re-)emission \citep{Rubincam2000}.

Understanding the rotation properties of asteroids has long been of interest to astronomers \citep[e.g.][for early work into the limits of rotation period and the tumbling nature of some small bodies]{Weidenschilling1981,Harris1994}.
This is still at the forefront of research, with more lightcurves being published year-on-year \citep{Harris2015}.
This will only increase with more large survey telescopes coming online soon, such as the Legacy Survey of Space and Time \citep[LSST][]{LSST2019}, the number of new asteroids that are well characterised will increase by more than 2 orders of magnitude on the Sloan Digital Sky Survey \citep[SDSS][]{York2000} which had $\sim 88000$ objects \citet{Parker2008}.
LSST needs a dedicated pipeline for asteroid classification, the Solar System Notification Alert Processing System \citep[SNAPS][]{Trilling2023} has been developed using ZTF data and is can scale up handle volume of data coming from LSST.
Work is being done to understand the biases of how  asymmetric shape high amplitude variations can lead to selection effects of such large sample of asteroids \citet{Levine2023}.
For the smallest objects ingenious observation techniques are employed to measure rotation periods such as intentionally streaking the asteroid down a column of pixel on a detector, and making the lightcurve from the streak, as done by \citet{Bolin2023}.

Because of such a rich history of study, many asteroids have a known rotation period.
These are catalogued by \citet{Warner2009} %TODO updated in citation http://www.MinorPlanet.info/php/lcdb.php
in the Light Curve Database (LCDB).
%// TODO fig of period vs diameter.   
In the October \nth{1}, 2023 release, the LCDB has nearly 9000 objects with periods determined to a quality code $U\geq 2-$, which is the minimum reliability recommended by \citeauthor{Warner2009} for statistical analysis.
This includes other TESS data from \citet{Pal2020} and \citet{Woods2021}.
The frequencies against diameter plot of these asteroids in \autoref{Fig:FreqVsDiam} shows the collection of all the asteroids in the LCDB.
The diameter presented in the LCDB here calculated from the absolute magnitude $H$ of the bodies, assuming an albedo.
\citeauthor{Warner2009} acknowledge that there are direct measurements of diameter for more asteroids in the present day, but continue to present the $H$ derived diameters for consistency.
The spin barrier \citep{Pravec2000} is seen in the red dashed lines as there are very few asteroids larger than $\qty{0.15}{\kilo\metre}$ rotating faster than $\sim \qty{10}{\per\day}$.
It is thought that the YORP effect spins up asteroids so fast that they pass above this barrier and then no longer have the internal strength necessary to hold themselves together, impling that all asteroids made to spin faster have torn themselves apart and that they have no tensile strength, i.e they are rubble piles instead of monoliths \citep{Harris1996}. This idea of the cohesive strength of small bodies is still explored today \citep[e.g.][for V-type asteroids]{Oszkiewicz2020}.
Comparing the periods I find of asteroids that are already in the LCDB is key to making sure my methods are reliable, and that the periods for asteroids for which no data exists can be trusted.

\begin{figure}[t!]
  \centering
  \includegraphics[width=0.7\textwidth]{../Test Code/Testing Figures/Diam-FreqPlot.pdf}
  \caption[Asteroid frequencies vs diameter (LCDB)]{
  Rotaion rate of asteroids by their size, recreated from the  October \nth{1}, 2023 LCDB after \citet{Warner2009}.   
  }
  \label{Fig:FreqVsDiam}
\end{figure}

\subsubsection*{Interstellar Objects}
%Omuamu
High amplitude variation has come to the forefront of questions about asteroid properties because of the first interstellar object, 1I/\omuamua \citep[see][for a review]{Bannister2019}.
\omuamua was determined to be spectroscopically red \citep{Fitzsimmons2017, Meech2017}, and having a photometric colour in the neutral end of the solar system range \citep{Bannister2017}.
1I was classed as asteroid due to lack of a coma, and no noticeable activity.
This contrasts the second ISO, 2I/Borisov, which had many cometary characteristics \citep[see ][for a review]{Dorofeeva2023}. %\\TODO cite, as opposed to 2I??
\omuamua was measured to have a rotation period of \qty{8.67(0.34)}{\hour} \citep{Belton2018} and seemed to be tumbling \citep[e.g.][]{Drahus2018,Fraser2018}.
Combing the tumbling with an elongation ratio of up to \qty{6(1)}{}:1 \citep{McNeill2018}, 1I is said to have a cigar shape \citep{Belton2018}.
The peak to peak amplitude variation of this ISO was \qty{2.5}{\mag} \citep{Meech2017} over its double peaked light curve.
This is quite interesting, as it varies more than most asteroids.
With the full sky survey of bright asteroids, we hope to find many asteroids with such a large amplitude variation, and to see just how rare \omuamua is.


\subsection{TESS}\label{SubSec:TESS}

%Why TESS
TESS is a large area, high cadence imaging, space telescope  \citep{Ricker2014}.
TESS is tasked with observing one piece of sky for \qty{27}{\day} at a time (a sector), delivering \qtyproduct{96 x 24}{\degree} full frame images (FFIs) at regular intervals.
These FFIs are built from stacked \qty{2}{\second} exposures, leveraging the short readout times of the 16 CCD cameras on-board.
With the initial cadence for these full frame images set to \qty{30}{\minute}, the time resolution of TESS is unparalleled.
The highest frequency (shortest period) that can be found from time series data like this is at the Nyquist limit, twice the minimum time between images.
Due to the short cadence, the Nyquist frequency of TESS is quite high, \qty{1}{\per\hour}.
This is well sampled enough to characterise most variable stars, as well as find the orbital periods of exoplanets to a high precision.
As the mission was extended, after TESS had already mapped the entire sky, the length of the FFIs has decreased to \qty{10}{\minute} and then later to \qty{200}{\second}.
This decreased in the FFI exposure time comes with a complementary increase in the Nyquist frequency.
This high time resolution and observation area does come at the cost of spatial resolution, as the pixels are each \qty{21}{\arcsec} square.
Of interest here is what such a high sampling rate can do for statistic on the asteroid population.
For bright asteroids the rotation periods should be able to be easily determined from this vast dataset.
The shortest FFIs will be able to accurately determine the rotation periods of all but the fastest rotating asteroids, most of which will be too dim to see in the TESS data.

There have been attempts to find and classify the asteroids in TESS data before.
\citet{Pal2018} proposes that TESS will be a good instrument for solar system object study, and simulate some detections down to \nth{20} visual magnitude, but that good photometry should only be expected to $V \lesssim 19$.
The sentiment of using TESS in this way is shared by \citet{Wong2019}.
The first data release in \citet{Pal2020} catalogues nearly 10,000 asteroid light curves from the first year of TESS operation, and they report that they triple the number of asteroids with accurate rotation periods.
\citet{McNeill2023} also analyse TESS cycle 1, and detect almost 38,000 objects.
They determine reliable rotation periods for about 3,500 asteroids in this sample and show a lack of reliability for objects with periods less than \qty{3}{\hour}.
In the overlap from the data from \citet{Pal2020}, \citeauthor{McNeill2023} find good agreement between the two sets of periods and amplitudes.
The Minor Planet Center (MPC) gets regular updates from TESS thanks to the work of \citet{Woods2021} and their \texttt{LINEAR-TESS} program, which creates tracklets of objects over a day, and then stitches these together to form a track of each asteroid through an FFI.
\citet{Gowanlock2024} uses TESS photometry as well as  observations by the Zwicky Transient Facility \citep[ZTF, ][]{Bellm2018} to get a longer baseline on mutually observed objects while combining ground and space based observations.
There were not very many objects in both samples with known periods for comparison, with only 222 objects analysed, however they demonstrate that the technique is effective.
Fainter and unknown solar system objects can be found by shift and stacking \citep{Holman2019, Payne2019,Rice2020} or taking a fast X-ray transform \citep{Nguyen2024} of the FFI data.

This work aims to extend these studies to more sectors, hence characterising more asteroids.
The aim is to get a volume complete set of asteroid lightcurves and periods, by analysis of the whole sky as seen by TESS using data from more than just the first year of TESS's operation.
I also employ a different data reduction method, reducing the data with the \texttt{TESSreduce} package \citep{Ridden-Harper2021} should increase the accuracy of the photometry due to the improved background subtraction process.
Because of the survey properties, TESS provides a self-consistent way to measure the properties of asteroids over the full sky.
By using the higher cadence FFIs of later TESS sectors, the $P\leq\qty{3}{\hour}$ limit on the accurate periods \citep[as found by ][]{McNeill2023} should be lowered.
Another beneficial part of this work is that as part of a full sky transient survey using TESS, \texttt{TESSELLATE} \citep{TESSELLATE}.
Asteroids can spike the brightness of a pixel for only a few frames while it is passing by.
This can confuse pipelines looking for  transients such as flare stars and supernova, as they can look very similar, having the brightness change in one pixel for a short time.
The goal of finding all the asteroids will allow for the removal of these spikes from the transient pipeline, as well as to understand the asteroid population better.
By filtering the asteroid detections out, while also self consistently determining rotation properties of a large sample of the population, more science can be done with the one action.


\section{Methods}\label{Sec:Meth}

\subsection{Querrying Databases}\label{SubSec:Querry}

To check for asteroids in the TESS data, the positions of the asteroids with time are required.
For most asteroids, their orbital elements are well known, so it is a matter of looking them up and cross-matching with transients in the TESS data.
Python was used to make API calls to
%{Skybot}\footnote{\href{https://vo.imcce.fr/webservices/skybot/}{Skybot: https://vo.imcce.fr/webservices/skybot/}} 
Skybot \citep{Berthier2006} to get positions of asteroids in a cone search box in right ascension (RA) and  declination (Dec) space.
The search was limited to only asteroids with an average visual magnitude $V\geq \qty{20}{\mag}$.
As TESS sectors are \qty{27}{\day} long, so querying every \qty{12}{\hour} is both computationally feasible and allows for a good understanding of a bodies motion. %TODO cite numpy
These positions are still very sparsely spaced in time compared to the TESS data, so an interpolation is needed to bridge the gap.

Knowing where the asteroids are is helpful for finding them in the archival TESS data, but knowing more about the individual asteroids is also useful for population statistics.
The \texttt{astroquery} \citep{Ginsburg2019} package was used to query JPL Horizons% {JPL Horizons}\footnote{\href{https://ssd.jpl.nasa.gov/horizons/}{JPL Horizons: https://ssd.jpl.nasa.gov/horizons/}} 
was used to obtain the orbital elements of each asteroid, $a$, $e$, and $i$, as well as an absolute magnitude $H$, which is a good proxy for size (as discussed above). %TODO ref



\subsection{Interpolation}\label{SubSec:Interp}


\begin{figure}[h!]
  \begin{subfigure}[t]{0.5\textwidth}
    \centering
    \includegraphics[width =\textwidth]{../Test Code/Testing Figures/interpPos\_22\_2\_3\_4.pdf}
    \caption[Positions]{Positions of asteroids in a subsection.}
    \label{Fig:interpPos}
  \end{subfigure}
  ~
  \begin{subfigure}[t]{0.5\textwidth}
    \centering
    \includegraphics[width =\textwidth]{../Test Code/Testing Figures/1989 BC1PosCheck.pdf} %TODO one with more points???
    \caption[Errors]{Error in interpolated positions}
    \label{Fig:errPos}
  \end{subfigure}

  \caption[Interpolated positions and error of asteroids]{
    (a)
    The interpolated positions of asteroids in one subsection of a TESS sector.
    The colours of the lines are time sequenced, as shown in the colour bar.
    The alpha of the colours are scaled by the absolute magnitude $H$ of the asteroid, queried from JPL Horizons.
    Both celestial (RA and Dec) and ecliptic (ecLon and ecLat) coordinates axes are shown.
    The green $+$ is the centre of the subsection.
    This sunsection was chosen because of how many asteroids there were, enough to see the shape of the box, but not so much that the lines are overwhelming to the eye.\\
    (b)
    The difference in the interpolated position of an asteroid from bulk queries to JPL Horizons, with interpolated coordinate being subtracted from the coordinate returned by Horizons. The colour bar is a different colour map to (a) as the range of time is shorter. The extent of the box is one TESS pixel, \qty{21}{\arcsec}. This is of asteroid ``1989 BC1'', but all others checked are similarly accurate (It is very expensive to check them all by brute force queries of Horizons).
  }
  \label{Fig:interpandErrPos}
\end{figure}


With TESS data coming in $\frac12\unit{\hour}$ chunks, 24 interpolated points are needed between each API call.
These interpolated positions can be seen in \autoref{Fig:interpPos} for a subsection of sector from \texttt{TESSELLATE}.
There are a few interesting features, such as the asteroids are moving in the same direction, indicated by the colouring, they come in from low RA and Dec and tend to increase both coordinates as the month of the sector progresses.
There is also a large size range in this slice of sky, ranging over \qty{5}{\mag} in absolute magnitude, which can be seen in the alpha (or transparency) of the tracks in \autoref{Fig:interpPos}.
Because of a small change in position between each frame, \citep[$\sim \qty{1}{\px}$ per frame][]{Pal2018}, this interpolation should be accurate.
The
Checking against a higher frequency query to the JPL Horizons ephemeris  confirmed this accuracy on a few asteroids, as shown in \autoref{Fig:errPos}.
This is not a feasible way of getting all the positions, as the queries are per name and time, not per area and time like the cone search in Skybot.
They are also rate limited, so while a few asteroids can be checked, filling the gaps between the Skybot half day queries for all the asteroids would take far too long.
For the faster TESS FFIs, more interpolated points are needed, but a smaller the change in position between each point will be.
Forcing the interpolations to be at the same time as the TESS frames, which are known thanks to \texttt{TESSELLATE} saving them out, simplifies the matching to detections and forced photometry of these points.


\subsection{Detection Matching}\label{SubSec:Match}

%Match to detections
Matching these interpolated positions to \texttt{TESSELLATE} detections is important to lower the unknown transient outputs of this pipeline.
Having interpolated their positions, the asteroids have a well sampled set of RA, Dec and time values of where they should be in the TESS data.
They should show up in a pixel for a small number of frames, of order $\sim1$.
This is the same as a lot of other transient events, a sharp rise in brightness and then disappearing quickly again.
The number of frames do change, type Ia supernova will brighten is a matter of a few hours and then dim for days, while stellar flares are of similar profile by a smaller max brightness and a correspondingly shorter decay time.
Asteroids are very short spikes, however these detection pipelines are robust. %? Clarinda begs to differ
These pipelines have already found the aforementioned supernova, stellar flares and variable stars, by matching to catalogues of those types of objects (Gaia, ZTF etc.).
The job of this work is to catch all the asteroids in the set of all the transients that are returned from the pipeline.
To do this, catalogue matching is in order.


Using the \texttt{KDTree} algorithm \citep{Maneewongvatana1999} as implemented in \texttt{SciPy} \citep{2020SciPy-NMeth}, the RA and Dec coordinates of the interpolated points and the detections can be compared and matched together.
Filtering this \texttt{KDTree} output by restricting the time between spatially coincident matches to less than \qty{0.1}{\day} stops any accidental matches in position from non-asteroid detections.
A cut is then made by smallest distance, in coordinate space, and any distance must be smaller than \qty{0.01}{\degree} (\qty{36}{\arcsec}), which is within 2 TESS pixels.
Then, if multiple detection match to the same interpolated point, the detection with the smallest distance will be taken.
An example of the matches found using this \texttt{KDTree} is shown in \autoref{Fig:RADecMatch}.
It is clear that \texttt{TESSELLATE} is not detecting every frame that an asteroid is in.
Detections in \texttt{TESSELLATE} are done on a per pixel basis, and asteroids only cross each pixel for a short number of frames.
Not every interpolated point gets a match, due to a variety of reasons, which culminate in there being not an extreme enough of a difference between frames at that pixel location.
Instead, there are bunches of frequently matched points, and then patches where there are few matches detected at all.

%TODO fig for agregate of all asteroids in a sector, to go in results 

\begin{figure}[]
  \centering
  \includegraphics[width =\textwidth]{../Test Code/Testing Figures/DetectMatchPos.pdf}
  \caption[Interpolated and Detected Positions]{The RA (right plot) and Dec (left plot) against the time of observation of the interpolated points (blue dots) or the matches to \texttt{TESSELLATE} detections (black squares). The gap at low times is from the TESS downtime in the sector, when it was passing close to earth.}
  \label{Fig:RADecMatch}
\end{figure}

\subsection{Ligthcurves}\label{SubSec:Lightcurves}

%light curves; detected VS forced interpolated 
There are two sets of points to take light curves from.
The matches from the detections, which already have a flux calculated, and the interpolated points themselves, which are more numerous but require forcing the photometry.
There were some challenges getting the flux of the interpolated points, even when \texttt{TESSELLATE} has already reduced all the FFIs of interest.

Using the \texttt{SkyCoord} module of \texttt{Astropy}, the world coordinates, in RA and Dec, can be transformed into pixel coordinates, $x$ and $y$, using a world coordinates system (WCS) from TESSELLATE.
\texttt{Photutils} \citep{Bradley2024} was then used to do aperture photometry with a \qty{1.5}{\px} radius to account for the spread in the TESS PRF. %?cite PRF

The integer coordinates of the asteroids seemed to be leading the object in any given frame of TESS. %?FIG?
This was decreasing the total flux of the asteroid, as the box used for aperture  was hitting mostly background.
This also caused a ``sawtoothing'' effect, where the lightcurve would jump from a minimum to a maximum when the integer coordinate snapped from being a pixel out to back on the asteroid.
Both of these issues can be fixed by calculation of the centre of mass (COM), the average brightness in $xy$ space on a $5\times5$ \unit{\px} box of the image centred on the interpolated position.
I didn't take the highest pixel value in this box because there could be a rouge star in the field that it snaps to, instead of the asteroid.
A \texttt{Photutils} aperture photometry centred on the COM with the same radius of \qty{1.5}{\px} sorted some of the discrepancies in the lightcurve, with a far smoother transition from maximum to minimum, as expected from rotating asteroids. Hereafter, unless otherwise specified, ``flux'' refers to this COM flux measurement of the asteroid.


%TODO data cleaning explaination
A few steps need to be taken to clean these COM lightcurves. 
Following \citet{McNeill2023}, any fluxes more than $3\sigma$ from the mean value are sigma-clipped from the forced photometry lightcurves.
This does introduce gaps into an otherwise regularly sampled lightcurve, but the contamination from background stars the object moves through of is too great.
The subsections made by \texttt{TESSELLATE} have a small overlap. 
If an object moves between these, there will be a few frames where it is found in multiple of these subsections.
This makes the lightcurve have two simultaneous observations of the asteroid, which clearly is not possible.
At the very edge of the subsections, the asteroid can have a COM position that would have the aperture be summing over pixels that are outside the field, and thus the asteroid would have a lower flux value on the edge.
The overlap is great enough that the decrease in flux should only be happening on one of the edges at a time.
So, sorting by flux on a per asteroid basis, then removing the duplicate observation times, keeping only those with the highest flux, will clean the lightcurve. 


\begin{figure}
  \centering
  \includegraphics[width =\textwidth]{../OzData/SingleBodyLCUlyssesCorrectedAgain.pdf}
  \caption[Light curves Example]{The light curves of the asteroid (5254) Ulysses throughout its entire journey across sector 29, with the flux of the interpolated positions in blue circles and the flux from the COM in orange diamonds, and the \texttt{TESSELLATE} flux for any matched points in the black squares.
    The Flux axis is measured in counts, as calculated by \texttt{TESSreduce}, and the times are in modified Julian date (MJD).}
  \label{Fig:FullSingleLC}
\end{figure}
A comparison between the three light curves is interesting, as one might expect only the brightest parts of the lightcurve will be matched.
\autoref{Fig:FullSingleLC} shows these three light curves for a single body.
The COM flux seems much more stable than the interpolated positions flux on its own, and at a higher average flux count.
This should result in more accurate periodograms as there is less overall curve to the data.
The matched points are comparatively sparse for this object, which is not the case for all.
Many have no matches, and some have almost fully matched, this will be discussed more later.
Lightcurves with an average flux $<10$ counts are lost in the random noise floor, so a minimum mean flux limit needs to be applied to ensure quality and reliability of any periods calculated.


\subsection{Period detection}\label{SubSec:Periods}

%Periods and amplitueds for everything detected

The next part of my analysis has to do with determining the periods and amplitudes of each asteroid's light curve.

A widely used technique in determining the periods of astronomical data is the Lomb-Scargle periodogram \citep[\citet{Lomb1976,Scargle1982}, but see][for a review]{VanderPlas2018}.
This method can efficiently take discrete time-series data of which the observations are unevenly spaced and calculate the frequencies present in the sample.
TESS should have rather evenly spaced data, as the FFIs are produced on a set cadence, but sometimes data will be missing.
For example, it may have been sigma clipped due to a spike in brightness that is unphysical for an asteroid, or TESS sees the same asteroid in the same sector on either side of the day long mid-sector break it takes for data transfer.
The periodogram reduces to both a fast Fourier transform or a direct least squares fit in appropriate limits, but outperforms them markedly in unevenly sampled data.

There are a few different implementations of a Lomb-Scargle periodogram available in world of astronomy python packages.
I first tried using the \texttt{Lightkurve} \citep{Lightkurve2018} package built for period analysis of TESS (and Kepler) data of variable stars.
\texttt{Lightkurve} was easy to use but didn't allow for fine-tuning of the periodogram, or give any model of the curve it produced.
The next option was peeling back a layer of abstraction and using the Lomb-Scargle periodogram as implemented by \texttt{Astropy}\citep[\citet{Astropy2022} but see][for the implementation]{Vanderplas2012,Vanderplas2015}.
There were interesting similarities and differences between the packages.
\texttt{Lightkurve} is based on the \texttt{Astropy} methods, but does not give all the functionality, instead opting to simplify the process.
\texttt{Astropy} gives a statistic on how confident it is in the detected frequency is, the false alarm probability.
It also returns a model, in the form of

\begin{equation}
  \label{Eq:LCModel}
  F(t;f,\vec{\theta}\,) = \theta_0 + \theta_1\cdot\sin{2\pi ft} +\theta_2\cdot\cos{2\pi ft}
\end{equation}

where the flux, $F$, is given in terms of time, $t$, the calculated frequency, f, and the model parameter $\vec{\theta} = [\theta_0, \theta_1,\theta_2]$.
These parameters are useful in reconstructing the sinusoid without rerunning the periodogram.
The method used for the periodogram fits is the new \texttt{nifty-ls} package \citep{Garrison2024} that has can be implemented on its own, or by interfacing with \texttt{Astropy}.
I use the later implementation as the speed-up from the non-uniform fast Fourier transform is helpful on the large dataset from \texttt{TESSELLATE}, but the \texttt{Astropy} interface is how I had everything set up to run on just the subsection I was trialling my code on.

The minimum period returned by the periodograms will be set by the Nyquist limit of the data. 
For the \qty{10}{\min} FFIs (the cadence analysed here), the Nyquist period is $\qty{20}{\min}=\frac13\unit{\hour}$, a threefold increase in time sensitivity for fast rotating asteroids.  
Based on their experimentation with the \qty{30}{\min} cadence FFIs, \citet{McNeill2023} use a \qty{17}{\day} maximum period for their periodogram analysis.
While a lack of signal in the rotation curve causes the calculated period to find this low frequency limit, such cases can be filtered out by keeping only periods less than \qty{90}{\percent} of this maximum. 
They recommended that periods less than \qty{3}{\hour} be interpreted with caution, due to lack of reliability in recovering injected sources.
Because of the increase in FFI rate, this time will be droped to \qty{1}{\hour} for the minimum reliable period, as the lightcurve sampling on these timescales is equivalent.

%???
There will be differences in the period between the matched points and the interpolated points, not just because of the difference in average flux, but also from the larger number of points. 
The matches are often having fewer observations and are much sparser in time that the forced photometry measurements (see the black squares in \autoref{Fig:FullSingleLC}), so would give worse periodograms.
While the rotation period and other properties of the lightcurve will be calculated from the forced photometry on the COM positions, the matching is still beneficial to the \texttt{TESSELLATE} pipeline as a whole.
Taking previously unidentified detections and assigning them to an asteroid category, the remaining unknown transients can be more efficiently searched for other interesting astrophysical phenomena. 


\begin{figure}[!t]
  \centering
  \includegraphics[width=\textwidth]{../OzData/PeriodogramUlyssesUpdated.pdf}
  \caption[Periodogram example]{\textit{Upper Panel}:The Periodogram of the COM flux of an asteroid, with the power of each frequency displayed, both axes are in log scale. 
  The maximum power of the periodogram is highlighted with the gold star. 
  This maximum is the calculated rotation frequency
  The known frequency, from the LCDB \citep{Warner2009}, is shown in the blue dotted lines, and a factor of 2 alias each way is shown in the purple dotted lines ($2*$) and red dotted lines ($0.5*$). 
  \textit{Lower Panel}: The COM flux lightcurve and the returned model (dashed black curve) of the lightcurve from the periodogram analysis. The frequency of this model is the same as the gold stared point in the upper panel.
  }
  \label{Fig:PeriodEx}
\end{figure}


An example of such a periodogram and the corresponding fit to the lightcurve is given in \autoref{Fig:PeriodEx}. 
The periodogram calculated that this object had a frequency double that listed in the LCDB, as seen with the vertical dotted lines. 
This is expected due to the double peaked nature of most asteroid lightcurves \citep{McNeill2023}.  
The stared peak is by far the most prominent peak in the periodogram, as it is in log scale.
Not all the asteroids have such well-defined maximums, and the peak power of the periodogram is an important factor in its quality.
A peak power less than 0.1 means there is little to no signal observed and it will be almost indistinguishable from the noise.
The lower panel of \autoref{Fig:PeriodEx} shows the model given in \autoref{Eq:LCModel} fit to the COM flux lightcurve by \texttt{Astropy}.
This looks to be a good fit, with the peaks and troughs lining up.
The amplitude of the sinusoid appears to be slightly lower than the variation of the data, but it is consistent with the scatter.
This model is successfully fit over a break in the lightcurve, and ignores the observations with a lower flux that did not quite get sigma clipped.
This asteroid was chosen as an alias of the known frequency is recovered by this analysis, which is promising for those asteroids where no known frequency exists. 
The double alias is expected here due to the double peaked nature of the lightcurve \citep{McNeill2023}.   
%// TODO discuss fit of model...

%// TODO single object periodogram
%Maybe a bad one too

\subsection{Asteroid Statistics}\label{SubSec:StatMeth}

%Run on server
The \texttt{TESSELLATE} pipeline has been running on the OzSTAR supercomputing facilities.
After I was confident that all the parts of the asteroid detection and subsequent light curve analysis worked as required on one subsection of one camera of one CCD of one TESS sector, the same code was refactored slightly to work on OzSTAR and a large-scale analysis of all the processed TESS sectors can be run. I will be looking for completeness of detections of asteroids, as well as accuracy of periods and amplitude variation.

For accurate statistics, accurate data is needed. 
So there must be checks on the quality of the values calculated. 
These have mostly been discussed already, but for completeness I list them here.
A lightcurve must have an mean flux of at least 10 counts to be reliable.
For an accurate periodogram, it is required that the lightcurve also contain more than 200 observations, which is very easy for TESS and its \qty{10}{\minute} cadence.
The periodogram itself must have a peak power of at least 0.1, and the calculated period must be $\qty{1}{\hour}\leq P \leq \qty{367.2}{\hour}$, where the upper bound is \qty{90}{\percent} of the \qty{17}{\day} window suggested by \cite{McNeill2023}.  
\begin{figure}[t]
  \centering
  \includegraphics[width=\textwidth]{../OzData/LCBDcompLogScaleLineText1.pdf}
  \caption[Known against calculated period]{A comparison of the calculated periods of the objects in the Sector to their known periods in the LCDB. 
  Dotted lines of a constant relationship are plotted, coloured the same as \autoref{Fig:PeriodEx}.
  Both axes are in log scale.
  }

  \label{Fig:LCDBcomp}
\end{figure}
While some of the checks were chosen to agree with the past work of others, some were chosen to make sure the derived periods were in agreement with the known ones from the LCDB.
A comparison between objects in both the sector and the LCDB is seen in \autoref{Fig:LCDBcomp}. 
The $y=2x$ line is of interest here, the known periods being twice the detected period. 
This line corresponds to the calculated frequency being at the double alias of the known frequency, as frequency and period are inveses of each other. 
As seen in \autoref{Fig:PeriodEx} and expected from \citet{McNeill2023}, this line is what the periodogram should return. 
Most of the asteroids the pass the cut line along or near this key relation. 
Few asteroids on the line fail the checks, which is also reassuring that the checks do what they are supposed to and filter out incorrect periods.
The vertical line of grey points on the right-hand side of \autoref{Fig:LCDBcomp} is on that maximum of the window of searched periods at $\qty{408}{\hour}$, and demonstrates why the \qty{90}{\percent} upper bound is necessary.
Many asteroids, across an order of magnitude of know periods, have detected periods at the edge of the window which shows that the periodogram will return this value when no signal is present.  
There are about a dozen asteroids that pass the checks with large detected periods when compared to their known period. These green points in the lower right of \autoref{Fig:LCDBcomp} speak to the checks not being perfect. 
Some experimentation was done on the limiting values, and the ones presented here achieved the highest recovery of asteroid periods, while minimising the number that were incorrectly recovered. 
The idea of doubleing the detected period of all the asteroids and calling that number the rotation period has had some doubt cast on it, as there are a couple of points recovered nicely on the $y=x$ line, indicating that the true period is found. 
Either the LCDB has not accounted for the alias, which is unlikely as I only took the periods with high quality codes as they suggest \citep{Warner2009}, or the rotation signal sometimes returns the true period.
The latter implies the lightcurve is not double peaked, as is normally expected.


\section{Results}\label{Sec:Res}

This is the results for TESS sector 29.
This sector was chosen because it is early in the \qty{10}{\min} cadence of TESS FFIs, and had been fully run through TESSELLATE.
Similar results can be found in any of the sectors, the computation time was a limiting factor in getting more sectors processed.

\subsection{Asteroid Properties}\label{SubSec:AstPropRes}

The orbital element properties of the asteroids were returned by JPL Horizons queries.
A plot of the semi-major axis against the other orbital elements, eccentricity and inclination, is shown in \autoref{Fig:AEI}.
Looking at the data as a whole, all objects in the field of the instrument during the sector, with $V \leq \qty{20}{\mag}$, are plotted.
The exception being comets and centaurs, which have an $a$ (really $q$, the perihelion distance) too larger to see the detail of the majority of the objects. 
The distinction between the resonances in the main belt can be seen.
There are vertical strips separating the middle belt from the inner and outer belt.
The Jupiter Trojans are well-defined, at just past \qty{5}{\au}, with a small range of eccentricity but a large spread in the inclination.
A some Mars crossers and near earth asteroids (NEAs) at progressively smaller $a$ and a few more objects that are scattered with high $e$ and/or $i$ complete the distribution.
Many objects have failed the quality checks on their lightcurve and periodogram, but the ones that pass have a similar distribution in orbital parameters to the sample as a whole.
In total, 309 objects passed all the quality checks, out of a total sample of 5664 small bodies in the sector.


\begin{figure}
  \centering
  \includegraphics[width=\textwidth]{../OzData/AEIplotqualCut.pdf}
  \caption[aei distribution]{
    The $a$ against both $e$ and $i$ of all the objects that \texttt{SkyBot} found in Sector 29 with $V <\qty{20}{\mag}$(Other than the comets and centaurs with poorly defined $a$ values).
    The grey crosses are those objects that did not pass the quality checks on the lightcurve and periodogram, while the green is those that did. 
    The values of these proper elements were queried from JPL Horizons on 30/09/24.
  }
  \label{Fig:AEI}
\end{figure}


The number of asteroids belonging to each class of objects is highlighted in \autoref{Fig:NumPerClass}. 
As expected from the orbital elements in \autoref{Fig:AEI}, the majority of the minor planets fall in the three regions of the main belt. 
There are some objects that are not asteroids, 20 comets were in the field and 1 centaur.
While they have a V mag less than 20, these objects are often not bright enough to have lightcurves with high enough mean fluxes to be detected well by TESS, as seen by the green bars.
Around \qty{5}{\percent} of any class have lightcurves and periodograms that pass the quality checks. 
The highest fraction is the $\frac29$ of the Amor family of NEAs, however with only 9 total objects the sample is too small to be representative of the detectability of this class as a whole.


\begin{figure}
  \centering
  \includegraphics[width=\textwidth]{../OzData/classesBarqualCut.pdf}
  \caption[Class distribution]{A bar chart distribution of the total number of objects per class that \texttt{SkyBot} found in the Sector (blue) and the same distribution of those that passed all the quality checks (green). 
  The numbers above each class (coloured the same) correspond to the height of the bar, as some classes have too few objects to be seen.
  As with \autoref{Fig:AEI}, this is for all objects brighter than \nth{20}\unit{\mag}.
  For the class labels, MB is ``main belt'' and ``NEA'' is near earth asteroid.}
  \label{Fig:NumPerClass}
\end{figure}

\subsection{Detectability}

\begin{figure}
  \centering
  \includegraphics[width=\textwidth]{../OzData/recoverdHistBkgLimof10AtanBothLegendFixed.pdf}
  \caption[Recovery plot]{A histogram of known average visual magnitude of each object against the whether the object has a lightcurve with an average flux value above 10 %!CHECK
  counts for its COM flux (blue triangles) or passes all of the checks on the lightcurve and periodogram (green dots).
  A fit of an $\arctan$ function is provided in dotted lines of the same colour. 
  }
  \label{Fig:RecovPassorFail}
\end{figure}


% \begin{figure}
%   \centering
%   \includegraphics[width = \textwidth]{../OzData/matchesHist29.pdf}
%   \caption[Magnitude of matches]{The visual magnitude of the asteroid against the fraction of asteroids in this bin with more than 3 %!CHECK
%     matches.
%   }
%   \label{Fig:MagofMatches}
% \end{figure}


\begin{figure} %TODO fix.
  \centering
  \includegraphics[width=\textwidth]{../OzData/pointsMatchesNumberHistdouble1.pdf}
  \caption[Number of observations or matches]{\textit{Left}: A histogram of the number of observations in the lightcurve of each asteroid (orange solid line) in 100 observation bins. \textit{Right}: The number of matches each asteroid gets in the \texttt{TESSELLATE} detections (black dashed)in 100 match bins.
  Both are limited to 600 asteroids on the y-axis, which pales in comparison to the number of asteroids with $\leq$ 100 matches and yet is too large for detail to be seen in any higher bins. }
  \label{Fig:MatchInterpHists}
\end{figure}

\subsection{Period Analysis Results}\label{SubSec:PerRes}

%TODO amp vs period

\begin{figure}
  \centering
  \includegraphics[width = \textwidth]{../OzData/QualPervsAmp1.pdf}
  \caption[Amplitude of Variation against Rotaion Period]{The amplitude of variation in the model from the period analysis against the calculated rotation period.
  This is for all 309 %!CHECK
  objects that pass the quality checks.
  }
  \label{Fig:perVSamp}
\end{figure}


\section{Discussion}\label{Sec:Disc}

\subsection{Comparison to other work }%\citet{McNeill2023}}

%TODO redo fig 1??

%! IF Matching works both to M23 and LCDB !!!

\subsection{Future work}

quality checks aren't great

run on rest of TESSELLATE data

\section{Conclusion}\label{Sec:Conc}

It works ok

but could be better

summerise findings

\newpage %* Page number above here must be <=25

\section*{Acknowledgements}
\addcontentsline{toc}{section}{Acknowledgements}
%OzSTAR asks for:
\small
This work was performed on the OzSTAR national facility at Swinburne University of Technology.
The OzSTAR program receives funding in part from the Astronomy National Collaborative Research Infrastructure Strategy (NCRIS) allocation provided by the Australian Government, and from the Victorian Higher Education State Investment Fund (VHESIF) provided by the Victorian Government.

This research has made use of data and/or services provided by the International Astronomical Union's Minor Planet Center.

This work has used data and/or services provided by NASA's Solar System Dynamics website, \url{https://ssd.jpl.nasa.gov/}. Specifically JPL Horizons: \url{https://ssd.jpl.nasa.gov/horizons/}.

This research made use of Photutils, an Astropy package for detection and photometry of astronomical sources \citep{Bradley2024}.

%*Bib
\addcontentsline{toc}{section}{References}
% \bibliographystyle{jphysicsB}
\bibliography{bibfile.bib}
\end{document}



